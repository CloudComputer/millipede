\documentclass{journal}
\usepackage{ textcomp }
\usepackage{ifpdf}
\usepackage{ifthen}
\usepackage{mathtools}
\usepackage{subfigure}
\usepackage{epsfig}
\usepackage{ amssymb }
\usepackage{amsthm}


\title{\LARGE \bf
Waterfall proof
}
\author{Chris Nicholls
}
%\institute{Oxford University Computing Laboratory}

\begin{document}

Define a graph as $G = (N,E)$
where $N$ is the set of nodes and $E :: N\times N \times \mathbb{N}$ is the set of weighted edges
that must satisfy the following properties:
\begin{enumerate}
\item Transitivity. $(a,b,k) \in E  \iff (b,a,k) \in E$, if there is an edge from node a to node b, then there is an edge from node b to node a of the same weight.
\item Uniqueness.   $(a,b,k_1) \in E\ and\ (a,b,k_2) \implies k_1 = k_2$, there is at most one edge between two nodes.
\item We also require that no two edges are of equal weight, $(a_1,b_1,k) \in E\ and\ (a_2,b_2,k) \implies a_1 = a_2\ and\ b_1 = b_2$. This restriction ??? provide details.

\end{enumerate}

Define a regional-minima to be an edge that is surrounded by edges of geater weight, i.e.
An edge $e = (a,b,k)$ is a regional minima iff
$\forall c \in E, l \in \mathbb{N} :
  (c,b,l) \in E \implies l \geq k\ and\
  (a,c,l) \in E \implies l \geq k $
Equivalently, $e$ is the edge of smallest weight conneted to nodes $a$ and similarly for node $b$.

??? example.

A path between to regional minima is then a sequence of $n$ edges $e_0 ... e_{n-1}$
such that $e_0$ and $e_n$ are the only two regional minima
and for each $ 0 leq i < n,\ , \exists a,b,c \in N \colon e_i = (a,b,w_1) , e_{i+1} = (b,c,w_2)$

\noindent An edge is path-maximal if it is the edge with greatest weight along a path.
The waterfall algorithm then elides all but those edges that are of maximal weight along some path.

\noindent Claim: An edge is path-maximal iff it is not the minimum edge away from either of its endpoints.

\begin{proof}:

\noindent `$\Rightarrow$'
\indent
  Let $e$ be an edge in the graph and suppose it is path-maximal.
  By definition there exists a sequence of edges $e_0 ... e_{n-1}$
  with $c = e_i$ for some $ 0 \leq i leq n$. Clearly $i \neq 0$ and $i \neq n$ since
  $e_i$ and $e_n$ must be regional minima. Therefore at each end of $c$ there is another edge
  leading from that node with weight smaller than that of $c$, so $c$ is
  not the minimum edge away from either of its endpoints.


\noindent
`$\Leftarrow$'
\indent
  Let $e = (n_1,n_2,w_e)$ be an edge such that it is not the minimum edge away from either of its endpoints.
  Let  $a_0$ be the minimum edge connected to $n_1$ and $b_0$ be the minimum edge connected to $n_2$

  Create a sequence of edges from $n_1$ by always picking the smallest edge from the current vertex until
  you the smallest edge is already in the sequence.
  Do  the same for $n_2$
  Show the endpoints of these sequences are regional minima
  Therefore e is path maximal.
  ....


\end{proof}
\end{document}
